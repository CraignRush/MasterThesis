\chapter{Discussion}
During the course of this thesis, motions of spheroids - such as beads or cells - over the \gls{gmr}-sensor have been modeled from a physics and signal processing perspective. In a magnetic field simulation it has been found that beads generate a well defined signal shape if the \glspl{mnp} are distributed homogeneously over their respective surface. Thereby, the size of each \gls{mnp} matters strongly. It is a strong trade-off between carrying enough magnetic momentum to get a sufficient amplitude in dipole superposition and at the same time not causing inhomogeneities in the magnetic signal. Nevertheless, a broad range from \SIrange{100}{1000}{\nano\meter} diameter of the labels showed a linear magnetization behavior above \SI{30}{\percent} of total coverage.\\
In the force-equilibrium simulation, a model for a rolling motion on the bottom of a functionalized microchannel was built on basic calculus estimations. Apparently, the dominant rolling forces act on a microbead in the \si{\nano\newton} range, which leads to a necessary interaction of single to thousand proteins according to their relative bond-release force per biomolecule. This could be subject to further refinements of the MATLAB framework with the differential equations for binding kinetics \cite{lit:bio:BindingModeling} or the rolling motion simulation more close to the reality. 

Those computational findings were adapted to \SI{8}{\micro\meter} diameter microspheres which were biotinylated or antibody-coated by carbodiimide chemistry in different surface densities. On the one side, the magnetic momentum of beads could now be varied through a saturation with streptavidin-\gls{mnp} conjugates and measured in the magnetic flow cytometer. On the other side, this allowed for a magnetic bead capture assay where a unspecifically functionalized channel prevented the rolling of beads with the matching ligand on their surface. Additionally, for a more robust measurement in future experiments, methods for the covalent surface modifications of glass, \gls{pdms} and \gls{sin} have been established and analyzed by fluorescence microscopy and an optical bead capture assay.\\
However, the covalent nature of the surface modification - or more precisely the non-physisorption - could not be revealed during this thesis. Also the covalent functionalization procedures have never been evaluated magnetically but only optically. A comparison between both measurements bears much room for misassumptions.

 

Start by briefly summarizing your major findings, but without repeating exact data from the Results. This makes your novel information clear to peer reviewers and, later on, readers. It also forces you to decide which findings you should focus on in the Discussion.
Thereafter, discuss possible underlying mechanisms. Why did you get these results, what is happening? Mechanisms, particularly molecular mechanisms, have very high “impact” in the natural sciences.

Next, compare your findings to those of other relevant publications and attempt to explain any discrepancies. If your findings disagree with those of others in the area, compare their publication to your own manuscript in minute detail, looking for any differences (especially in methodology) that might explain the discrepancy.

Consider the possible limitations of your own study—paradoxically, most reviewers consider an awareness and openness about potential weaknesses as a strength. However, do not forget to emphasize strengths as well.

Discuss the possible consequences of your observations and/or future investigations required or motivated. Be as concrete as possible about future perspectives. As in the abstract, writing the equivalent of “more research is needed” is meaningless—more research is always needed. Describe the hypotheses, questions or mechanisms that need to be investigated and/or methods that should be applied, concretely and concisely. And, of course, if your research findings have potential practical implications, discuss these in some detail as well.

Finally, state your conclusions—have you supported or rejected the hypothesis you posed, or obtained an answer to your research question?





eriving Navier-Stokes equation by the Cauchy momentum equation is complex and harbors several sources of error. First, an incompressible Newtonian fluid as well as channel boundary is assumed. The used water suspensions are approximated with negligible compressibility, which is not true for the real case. Also, for blood or other shear-thinning fluids these deviations are prone for high errors.  

Then, the divergence relation of the respective viscous stress (eq. \ref{eq:divergence_Stresstensor}) does not hold for non-uniform viscosity $\eta$.

For later studies in a matlab model, the flow velocity and shear stress computations were carried out with the error sources considered. 


--> signal analysis with wavelet analysis

Third, the transient term (eq. \ref{eq:navierstokes}) was neglected in all simulations, but a connected syringe pump possesses a slow rise time (Fig. \ref{fig:fluidic:pumpStability:transient}) and a remaining ``pulsation error'' in steady state (Fig. \ref{fig:fluidic:pumpStability:steadystate}). In effect, another error adds to the simulation, which is only valid after several ten seconds of the last flow rate change.
\begin{figure}
	\begin{subfigure}[b]{0.5\textwidth}
		\centering
		\addtocounter{subfigure}{1}  
		\subfigimg[clip,trim=115 100 80 60, height=100pt]{a} {./Ressources/Fluidic/Transient_SyringePump.jpg}		
		\addtocounter{subfigure}{-1}  
		\phantomsubcaption
		\label{fig:fluidic:pumpStability:transient}
	\end{subfigure}%
	\begin{subfigure}[b]{0.5\textwidth}
		\centering
		\addtocounter{subfigure}{1}  
		\subfigimg[height=100pt]{\textbf{b}}{./Ressources/Fluidic/SyringeSteadyState.eps}
		\addtocounter{subfigure}{-1}  
		\phantomsubcaption
		\label{fig:fluidic:pumpStability:steadystate}
	\end{subfigure}
	\capption{Syringe Pump error sources}{Set flow rate: \orangeline, Real Flow Rate: \blueline \subref{fig:fluidic:pumpStability:transient} Transient step answer of a syringe pump through a microtube with \SI{254}{\micro\meter} inner diameter. \subref{fig:fluidic:pumpStability:steadystate} Steady state flow rate error around the desired \SI{5}{\micro\liter\per\minute} dispensing rate. A sinusoidal behaviour caused by the microstepping can be observed. \cite{lit:fluidic:fluigentPumpStability}}
	\label{fig:fluidic:pumpStability}
\end{figure}


Second, the channel height varies in reality as a result of fabrication inaccuracies. 

modification of nh2 with paa and protein like cooh

Contact angle for silanization of surface methods more useful --> should be 1st approach for characterization

Anti-Biotin-PE working?
BNF-Dextran-Streptavidin unspecific binding?
electrostatic surface interaction
evidence covalent binding?

gas bubbles, adsprotion decrease, tubing exchange
activity of protein

bead rolling with biotin-cooh, biotin-nh2, plain-cooh, plain-nh2 --> velocity correction factor

concentration of beads in differential setup too high, maybe better results with lower.