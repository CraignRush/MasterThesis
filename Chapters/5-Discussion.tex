
\chapter{Conclusion}
During the course of this thesis, motions of beads or cells over the \gls{gmr}-sensor have been modeled from a physics and signal processing perspective. The magnetic field simulation targeted a quantitative statement about a measured cell's sensor signal for various \gls{mnp} sizes. Although small particles can be deposited in great abundance on a cell surface and are expected to cause in turn a very homogeneous signal, the resulting magnetic moment is small. \\
 In contrast, big \glspl{mnp} should generate a high moment but cause field inhomogeneities at the same time.  In simulations, it has been affirmed that beads generate a well-defined signal shape if the \glspl{mnp} are distributed homogeneously. Thereby, the size of each \gls{mnp} matters strongly. It is a severe trade-off between carrying enough magnetic moment to get a sufficient amplitude in dipole superposition and at the same time not causing inhomogeneities in the magnetic signal.\\
 At low coverages of the cell, the signal shape differs prevalently, most for the biggest particles. Nevertheless, a broad range from \SIrange{100}{1000}{\nano\meter} diameter of the labels showed a linear magnetization behavior above \SI{30}{\percent} of total coverage. Conclusively, the high divergence of great particles can be tolerated for cell concentration or overall detection measurements. In order to estimate morphological parameters or expression densities from the cell signal, smaller particles should be chosen predominantly.

In the force-equilibrium simulation, a model for the bead rolling motion on the bottom of a functionalized microchannel was built on basic calculus estimations. The dominant rolling forces on a microbead act in the \si{\nano\newton} range. Most dominant are magnetophoresis (\textasciitilde\SI{5}{\nano\newton}), Stoke's drag (\textasciitilde\SI{500}{\pico\newton}), and avidity (\textasciitilde\SI{150}{\pico\newton}, per molecule), which leads to a necessary interaction from a single to thousand proteins according to their specific release force per bond in order to cause surface adhesion. This could be subject to further refinements of the MATLAB framework with implementation of differential equations for binding kinetics \cite{lit:bio:BindingModeling} or the rolling motion to gain a more crisp representation of reality. 
\clearpage
%Vielleicht gibst Du hier etwas mehr der wirkenden Kräfte hinein. Gedanke war eine Abschätzung zu erhalten. Auch hier im Absatz eine kurze Einführung warum Du die Simulationen  machst. Hier kannst Du schon die Affintät reinbringen und die Bestimmung und Integration dieser in den magnetischen Workflow. Der Kontext fehlt hier sonst etwas. 
In absolute counting experiments, beads were diluted with buffer or whole blood and quantified in the magnetic setup. Empirical correction factors based on normalization and rolling velocity, to account for bead losses, were determined. To overcome these limitations, a differential counting setup was established and characterized for relative bead counting and time-of-flight applications. Optimized flow rates and magneto-resistances indicated a promising function range. However, the intrinsic bead losses and complexity of the construction faded this setup from the spotlight.

% Das ist hier so mittendrin. Was hältst Du davon den nächsten Absatz über diesen zu stellen. Das differentielle Zählen ist nett, aber am Ende des Tages nicht si signifikant wie die Punkte der Oberflächen. Für die Präsentation würde ich es in jedem Fall so machen. Das ist runder und Du schließt Ergebnissen ab, die in den Outlook einfließen können.

%Hier würde ich ebenso einführen wofür die Beads genutzt werden können. Proof-of-concept für die Belegung  (Expressionsanalyse von Zellen, das sollte nicht unter den Tisch fallen, da auch das ein sehr starkes Ergebniss der Arbeit ist und so noch nie gezeigt wurde für die magnetische Durchflussztyometrie) sowie Proof-of_Concept für Deine Oberflächen, um die Anzahl der Bindungsstellen definiert einzustellen.
The computational findings were adopted in a method to create different surface densities of biotin or antibodies on \SI{8}{\micro\meter}-beads by carbodiimide chemistry. The magnetic moment of beads could now be varied through saturation with streptavidin-\gls{mnp} conjugates and measured in the magnetic flow cytometer. 
Due to the underlying biochemical processes in essentially all experiments, some effects remain unexplained: First, fluorescent labeling of biotinylated beads with strept\-avidin-atto488 yielded robust and reliable results in the bead characterization measured by an optical flow cytometer. In contrast, a similar labeling process with a recombinant secondary-antibody Anti-Biotin-PE showed inexplainable, constant surface coverage and unspecific binding. This could very likely be related to a sterically blocked binding domain but remains a question to be solved.

Along with biotinylation, another effect is related to \gls{mnp} coverage of bio-functionalized polystyrene microbeads. Initially, several \glspl{mnp} from different suppliers (Ocean Nanotech, micromod) had shown a highly unspecific binding to the beads so that no biotinylation dependent effect was distinguishable from the background. This was confirmed by fluorescence microscopy and optical flow cytometry (Fortessa, BD Biosciences).\footnote{data not shown} Even so, an additional measurement with newly blended \gls{pbst} buffer instead of previously used \gls{pbs} with biotin-free bovine serum albumin or \gls{macs} yielded compelling data. Nevertheless, the buffer exchange may not be responsible for this behavior but rather bead ``age'' or the optimized functionalization protocol.
\clearpage
% Hier besteht viel Platz für Diskussion, welche Effekte sind verantwortlich (elektrostatisch wird angenommen, daher ist sicher der Block mit Tween sinnvoll, Beadalterung ebenso, aber das erklärt auch die unspezifischen Bindungen? MACS ist eine Wundertüte. Das Bindungsprotokoll wurde optimiert, aber welcher Ansatz genau kann den Unterschied gemacht haben, (ich habe den Teil der Results nicht parat, aber wenn dort nicht erwähnt, dann hier wo die Unterschiede lagen, das ist eine wichtige Info :)

For a more robust force measurement in future experiments, methods for the covalent surface modification of glass, \gls{pdms}, and \gls{sin} have been established. These modifications were based on different wet-chemistry oxidation techniques, with subsequent silane and carbodiimide chemistry. After bio-functionalization, they were then analyzed by fluorescence microscopy or an optical bead capture assay and showed a comparable surface quality to physisorption.

In contrast, a true covalent or more precisely non-physisorbed surface modification, as well as its projected advantages in stability and definedness, could not be demonstrated during this thesis. Also, the bead rolling assay on each chemically-treated substrate has only been evaluated optically but not magnetically due to the lack of sensors on the respective substrates.\\
Furthermore, the desired convenience of covalent bio-functionalization could not outperform simple, unspecific physisorption. In both cases, tedious handling with low shear inside a completely bubble-free microchannel together with the watertight exchange of microtubing is necessary. For covalently modified chips, in addition, \SI{12}{\hour} of laboratory work was required as auxiliary overhead while its superior stability was never confirmed. Therefore, it remains a supplemental procedure until an exhaustive characterization and further protocol optimization took place.\\
Besides, bio-functionalized surfaces have never been brought into contact with any physiological substance, such as blood or immune cells, which are very likely to disturb the system's operation.

Additionally, the existing magnetic flow cytometry platform has been extended and characterized for the use of affinity-based assays. With the help of a newly established model system for defined molecule densities on beads, their functional interaction in a microchannel led to the presented bead capture assays. For these, beads were immobilized in a laminar flow according to their respective biomolecular bond strength. To mediate their bonding to the channel bed, the microchannel was modified by physisorption of neutravidin. With these biologically augmented microfluidic systems, magnetic measurements showed a clear separation from fully- to sparsely-coated beads. Thereover, a dependency on the biotinylation-degree of microbeads has been measured with strong statistical support.
\cleardoublepage




%%
%Second, the channel height varies in reality as a result of fabrication inaccuracies. 
%
%modification of nh2 with paa and protein like cooh
%
%Contact angle for silanization of surface methods more useful --> should be 1st approach for characterization
%
%Anti-Biotin-PE working?
%BNF-Dextran-Streptavidin unspecific binding?
%electrostatic surface interaction
%evidence covalent binding?
%
%gas bubbles, adsprotion decrease, tubing exchange
%activity of protein
%
%bead rolling with biotin-cooh, biotin-nh2, plain-cooh, plain-nh2 --> velocity correction factor
%
%concentration of beads in differential setup too high, maybe better results with lower.