\chapter{Abstract}
Biosensors employ usually optical \cite{lit:bio:BioconjugateTechniques}, electro-chemical \cite{lit:fluidic:BindingPhysicsSurfaces} or magnetic \cite{lit:thes:reisbeck, lit:bio:MRCyte2016, lit:bio:NanoCytometer} transducers to detect biomarkers.  Optical  biosensors  have  proven high sensitivity.\cite{lit:Shapiro}  In turn, one of the  most  sensitive  optical  biosensors surpassed a detection limit of \SI{1}{\cfu\per\milli\liter} \textit{E. coli} with a time to turnaround of \SI{30}{\minute} by combining a flow  cytometer and fluorescent \glspl{mnp}.\cite{lit:flowCytometer} However, optical systems rely on the transparency and low optical background of the sample. The consequential preanalytics require specialized laboratory and expert personnel.\\
Magnetic flow cytometry targets these shortcomings under the trade-off, that until now lower throughput and less measurable parameters are limiting factors.\cite{lit:thes:reisbeck} Especially in optically dense samples, such as human body fluids and blood, magnetic flow cytometry can manifest its superior capabilities because of a negligible magnetic background in the biological samples.\cite{lit:bio:biochip:cd64} On the one side, this enables cheap point-of-care tests since the \gls{gmr} sensor costs less than \EUR{20} in comparison to optical spectrometers or cameras.\cite{lit:fluidic:HighFlowGMR,lit:bio:aflatoxinMNP,lit:bio:POC_CD64} On the other side, the sharp sensitivity and electronic speed of these sensors allow for precise single cell measurements.\cite{lit:bio:MRCyte2016,lit:fluidic:GMR_Quantification, lit:paperHelou} When magnetically labeled cells are rolling over such a sensor, information about a cell's size, morphology, and biomarker density can be extracted from a single signal pattern.\cite{lit:thes:michaelBauer} 

In this thesis, the existing magnetic flow cytometry platform has been extended and characterized for the use of affinity-based assays. With the help of a newly established model system for biomarker densities on cell surfaces, their functional interaction in a microchannel has been studied theoretically and experimentally.\\
First, the physical phenomena of rolling cells with a bio-functionalized surface under laminar flow conditions were be simulated from a hydrodynamic and magnetic point of view. Here, the synergy of fluid dynamics, inertia, viscous, and body forces was illuminated in an analytical force equilibrium.(\cref{sec:theo:force,sec:res:fluidSim,sec:res:forcesim}) Additionally, a numeric magnetic field simulation of \gls{mnp}-laden spheres was correlated to dipole signals in the magnetic flow cytometer.(\cref{sec:res:simMag})\\
Second, a reference system for the variation of surface receptor density was established and subsequently evaluated. With the developed methods, several proteins were attached to the surface of microbeads covalently. In the next step, the protocols were adopted to preserve a reliable dose titration of biotin and antibodies for the coating density.(\cref{sec:res:beadFunc}) \\
Third, absolute and differential concentration measurements were benchmarked for the magnetic flow cytometer. For absolute counting experiments, beads were diluted to exact concentrations with buffer or whole blood and quantified in the magnetic setup.(\cref{sec:res:ConcMeas,sec:res:wholeBlood}) As a result, empirical correction factors, to account for losses in syringe, connectors, and the parabolic flow profile, were determined.(\cref{sec:res:Correction}) To overcome these limitations, a relative counting apparatus has been engineered from two serial magnetic sensors. However, the complexity and poor robustness of the differential system requires further developments beyond this thesis.(\cref{sec:res:diffCounting}) Furthermore, the sensitivity of the sensor to weakly magnetized cells was evaluated in a pretest by \gls{mnp}-coating the previously functionalized microbeads.(\cref{sec:res:beadMangetiz}) \\
Ultimately, an affinity-based concentration assay will be presented which reveals promising results for the magnetic measurement of biomarker concentrations, exosomes or single cell surface proteins in the future.(\cref{sec:res:affinity}) For this purpose, beads were immobilized in a laminar flow according to their respective biomolecular bond strength. To mediate their bonding to the channel bed, the microchannel was modified by physisorption of neutravidin.(\cref{sec:res:physis}) Notwithstanding, the unspecific biofunctionalization dismantled swiftly. In turn, a covalent surface treatment strategy was researched, adopted to \acrshort{pdms}, \acrshort{sin}, and glass, and characterized by optical means.(\cref{sec:res:coval}) With these biologically augmented microfluidics, magnetic measurements showed a clear separation from fully- to sparsely-coated beads. Thereover, a dependency on the biotinylation-degree of microbeads has been measured with strong statistical support. Further insights into these effects could lead to a new segment of point-of-care tests and clinical assays.


