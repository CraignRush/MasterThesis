\chapter{Materials and Methods}

\section{Microfluidic Fabrication}

\subsection{Development of Layout}

\subsection{Patterning of Photoresist}

\subsection{Soft Lithography}
The fabricated wafer was placed the center of a \SI{90}{\centi\meter} petri dish. A \gls{abbr:pdms} mold was created by vigorous mixing of the pre-polymer base with its curing agent (Sygard 184, Dowsil) in a ratio of 10:1 (w/w). For \SI{3}{\inch} wafers, thin channels were casted from \SI{15}{\gram}, normal channels from \SI{20}{\gram} PDMS in the petri dish. Gas bubbles were removed from the mixture in a desiccator for \SI{20}{\minute} at \SI{2}{\hecto\pascal} , and the clear \gls{abbr:pdms} was cured in an oven (Um, Memmert) for \SI{1}{\hour} at \SI{60}{\degreeCelsius}. After curing, the \gls{abbr:pdms} mold was released from the petri dish carefully, taken off the wafer and stored in a clean petri dish upon further processing.

\subsection{Bonding of Microfluidic}
Under laminar flow, crosslinked molds were cut into pieces with the respecting single \gls{abbr:uF} with a razor blade. Holes for in- and outlet were punched through the containing channels with a biopsy puncher (ID \SI{0.5}{\milli\meter}, WellTech). The substrates and \glspl{abbr:uF} were sonicated in acetone and \gls{abbr:diH2O} for \SI{5}{\minute} and dried with filtered \gls{abbr:N2} completely. For the bonding of PDMS to various substrates different protocols have been established:

\subsubsection{PDMS Glueing}
Here, a micron-height layer of uncured \gls{abbr:pdms} was used as an adhesive layer between \gls{abbr:uF} and substrate. Approx. \SI{3}{\milli\liter} were poured onto a \SI{3}{\inch} wafer and spun down for \SI{5}{\minute} at \SI{6000}{\per\minute}. The microchannel was placed on the substrate by visual control of a stereo microscope (SMZ800, Nikon) with 8-fold magnification. Subsequently, the bonding process could be finished by a \SI{1}{\hour} bake at \SI{60}{\degreeCelsius} or over-night at room temperature.
\subsubsection{Plasma Bonding}
Due to the chemical nature of glass (or more generally oxides) and \gls{abbr:pdms}, the respective parts can be activated by the exposure to a controlled oxygen plasma which generates additional silanol (Si-OH) groups on their surfaces and removes impurities at the same time. Bringing the activated surfaces in contact triggers the formation of covalent bonds almost immediately. First, the acetone-wiped substrates and the microchannels were centered inside the plasma cleaner (Zepto, Diener). Second, vacuum was applied to a final pressure <\SI{0.2}{\hecto\pascal}. Third, the chamber was flushed with pure \gls{abbr:O2} until a chamber pressure from \SIrange{0.7}{0.8}{\hecto\pascal} had been stabilized. Fourth, the plasma process was executed with \SI{30}{\watt} (Power-Poti: 100) for \SIrange{45}{60}{\second} (Time-Poti: 15-20). Upon finish, the chamber was flushed for \SI{5}{\second} and ventilated. Immediately, the corresponding workpieces were brought into contact and pressed together gently. To ensure a durable bond, the assembled workpieces were baked for \SI{1}{\hour} at \SI{60}{\degreeCelsius}.

\begin{equation}
	Here goes the mass flow equation
\end{equation}

\subsubsection{Reversible Bonding}
To bond the \gls{abbr:uF} to a substrate reversibly and without residues, the channel can be brought into contact with the bottom part without any adhesinon agent. For low-pressure as well as vacuum driven flows, this method is preferrable due to its time and work efficiency.

\section{Surface Bio-Functionalization}

\subsection{Surface Activation}
To functionalize any surface with \ch{-OH}
\subsubsection{Hydrochloric-Sulfuric Acid Activation}


\subsubsection{Piranha Activation}

\subsubsection{Hydrofluoric Acid Activation}

\subsection{Chemical Surface Functionalization}

\subsection{Surface Bioconjugation}

\section{Magnetic Beadometry}

\subsection{Standard Parameters}

\subsection{Concentration Measurement}

\subsection{Whole Blood Bead Spiking}

\subsection{Bead Capture Assay}

\subsection{Optical Particle Tracking}

\section{Tensiometry}



