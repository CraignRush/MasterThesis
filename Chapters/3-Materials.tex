\chapter{Materials and Methods}

\section{Microfluidics and Sensor Fabrication}
The fabrication of a microfluidic channel on various substrates and sensor layouts 
\subsection{Development of Layout}

\subsection{Patterning of Photoresist}
\SI{3}{\inch} (100) silicon wafers (Si-Mat) were dehumidified in a drying oven (UN30, Memmert) for \SI{2}{\hour} at \SIrange{150}{180}{\degreeCelsius}. Then, immediately after they reached room temperature, they were placed centered inside a wafer spinner (WS-650-23B, Laurell Technologies). For the desired layer thicknesses \SIrange{2}{3}{\milli\liter} SU8-30XX (Microchem) were poured carefully onto the center of the wafer and the following program was carried out:
\begin{enumerate}[noitemsep]
\item \SI{500}{\rpm} for \SI{10}{s} at \SI{100}{\rpm\per\second}
\item \SI{3000}{\rpm} for \SI{30}{s} at \SI{300}{\rpm\per\second}
\item Ramp down at \SI{300}{\rpm\per\second}
\end{enumerate}
Upon finish, the wafer was gripped outermost with wafer tweezers and soft-baked on a hot plate (super nuova+, Thermo Scientific) for \SI{5}{\minute} at \SI{65}{\degreeCelsius} and at least \SI{10}{\minute} at \SI{90}{\degreeCelsius}. The optimal duration was determined if the gently touched resist did not stick to the tweezers. To prevent cracks in the resist caused by a fast temperature change, the wafer was cooled on the hotplate to room	temperature. Such processed wafers were stored for a maximum of \SI{4}{weeks} in a light-tight storage box.\\
To pattern the resist, the i-Line of a laser lithograph (Dilase 250, Kloe) was used. In preparation of the writing layout a AutoCADz \textit{*.dxf}-file with only one layer of polylines was imported to the program ``Kloe Design'', converted to contours and subsequently to polygons. For the filling a spot-size equivalent to the minimal structure resolution (as measured in \citet{lit:tech:rojda2020}) and an overlap of at least \SI{50}{\percent} was chosen. The writing trajectories were displayed for a last control before the export to ensure only closed contours. Finally the contour and filling were exported into separate files.\\
Both files were loaded in this order into the ``Kloe Dilase'' program. Also the preprocessed wafer was placed inside the laser writer and attached to the vacuumed stage. With the integrated camera the global zero was set to the wafer center by finding the horizontal or vertical edges and adding/subtracting the radius of the wafer (\SI{3}{\inch} $\approx\ \varnothing $  \SI{76.2}{\milli\meter}) intensities, writing speeds 


\subsection{Soft Lithography}
The fabricated wafer was placed the center of a \SI{90}{\centi\meter} petri dish. A \gls{pdms} mold was created by vigorous mixing of the pre-polymer base with its curing agent (Sygard 184, Dowsil) in a ratio of 10:1 (w/w). For \SI{3}{\inch} wafers, thin channels were casted from \SI{15}{\gram}, normal channels from \SI{20}{\gram} PDMS in the petri dish. Gas bubbles were removed from the mixture in a desiccator for \SI{20}{\minute} at \SI{2}{\hecto\pascal} , and the clear \gls{pdms} was cured in an oven (Um, Memmert) for \SI{1}{\hour} at \SI{60}{\degreeCelsius}. After curing, the \gls{pdms} mold was released from the petri dish carefully, taken off the wafer and stored in a clean petri dish upon further processing.  

\subsection{Bonding of Microfluidic}
Under laminar flow, crosslinked molds were cut into pieces with the respecting single \gls{uf} with a razor blade. Holes for in- and outlet were punched through the containing channels with a biopsy puncher (ID \SI{0.5}{\milli\meter}, WellTech). The substrates and \glspl{uf} were sonicated in acetone and \gls{dih2o} for \SI{5}{\minute} and dried with filtered \gls{n2} completely. For the bonding of PDMS to various substrates different protocols have been established:

\subsubsection{PDMS Glueing}
Here, a micron-height layer of uncured \gls{pdms} was used as an adhesive layer between \gls{uf} and substrate. Approx. \SI{3}{\milli\liter} were poured onto a \SI{3}{\inch} wafer and spun down for \SI{5}{\minute} at \SI{6000}{\per\minute}. The microchannel was placed on the substrate by visual control of a stereo microscope (SMZ800, Nikon) with 8-fold magnification. Subsequently, the bonding process could be finished by a \SI{1}{\hour} bake at \SI{60}{\degreeCelsius} or over-night at room temperature.
\subsubsection{Plasma Bonding}
Due to the chemical nature of glass (or more generally oxides) and \gls{pdms}, the respective parts can be activated by the exposure to a controlled oxygen plasma which generates additional silanol (Si-OH) groups on their surfaces and removes impurities at the same time. Bringing the activated surfaces in contact triggers the formation of covalent bonds almost immediately. First, the acetone-wiped substrates and the microchannels were centered inside the plasma cleaner (Zepto, Diener). Second, vacuum was applied to a final pressure <\SI{0.2}{\hecto\pascal}. Third, the chamber was flushed with pure \gls{o2} until a chamber pressure from \SIrange{0.7}{0.8}{\hecto\pascal} had been stabilized. Fourth, the plasma process was executed with \SI{30}{\watt} (Power-Poti: 100) for \SIrange{45}{60}{\second} (Time-Poti: 15-20). Upon finish, the chamber was flushed for \SI{5}{\second} and ventilated. Immediately, the corresponding workpieces were brought into contact and pressed together gently. To ensure a durable bond, the assembled workpieces were baked for \SI{1}{\hour} at \SI{60}{\degreeCelsius}.

\begin{equation}
	Here goes the mass flow equation
\end{equation}

\subsubsection{Reversible Bonding}
To bond the \gls{uf} to a substrate reversibly and without residues, the channel can be brought into contact with the bottom part without any adhesinon agent. For low-pressure as well as vacuum driven flows, this method is preferrable due to its time and work efficiency.

\subsection{Electrical Circuit}
Ground
PCB
Stacked PCBs with spacer
\subsection{Electronic Readout}
test,test
\subsubsection{Hysteresis Alignment}
test,test
\subsubsection{Single GMR}
test,test
\subsubsection{Dual GMR}
one MFLI supplies both at same freuqency. Aux Trigger tested, but no advantage.

\section{Magnetic Beadometry}

\subsection{Standard Parameters}

\subsection{Concentration Measurement}

\subsection{Whole Blood Bead Spiking}

\subsection{Bead Capture Assay}

\subsection{Optical Particle Tracking}


\section{Surface Bio-Functionalization}


\section{Tensiometry}


\subsection{Surface Activation}
To functionalize any silicon containing surface with \ch{Si-OH} groups which the utilized silane could interact with, multiple surface activation pathways were explored. First, substrates were cleaned in \gls{hcl}:\gls{meoh} and \gls{h2so4} before they were immersed in boiling water. Second, surface silanol groups were achieved by piranha immersion and third by a \gls{hf} dip.\\
For all methods, the following reagents were used: \gls{dih2o} (\SI{0,054}{\micro\siemens}, Merck MilliQ)), acetone (\SI{>99,9}{\percent}, VWR), \gls{etoh} (absolute, VWR), \gls{meoh} (\SI{99.8}{\percent}, VWR), \gls{acoh} (glacial, VWR), \gls{hcl} (\SI{37}{\percent}, Sigma-Aldrich), \gls{h2so4} (\SIrange{95}{98}{\percent}, VWR), \gls{h2o2} (\SI{30}{\percent} (w/w), Sigma-Aldrich), \gls{hf} (\SI{5}{\percent}, VWR)

\subsubsection{Work Safety Remarks}
Before the work with one of the acid solutions was carried out, serveral safety measures were implemented. As any diluted acid solution becomes very hot immediately due to the exothermic reaction, every container should be placed inside a cooled water or ice bath. Additionally, the beaker as well as concentrated acid flasks should be gripped firmly by a laboratory stand to avoid a tip over. As the reactivity of chemicals is highly temperature-dependent, the solutions was processed further when they had been cooled to \SI{<=70}{\degreeCelsius}. It should be also noted that - as in every chemical reaction, but especially ones with \gls{h2so4} - the acid was always poured into the other reactant to avoid splashing and boiling.

\subsubsection{Plasma Activation}

Hier plasma protocolle raussuchen

\subsubsection{Hydrochloric-Sulfuric Acid Activation}
To degrease any glass or \gls{sin} surface, a protocol according to \citet{lit:chem:Dressick} was used. There, the surfaces were first sonicated in acetone and \gls{dih2o}  for \SI{5}{\minute}. Afterwards these were immersed in a 1:1 (v/v) solution of \gls{hcl}:\gls{meoh} for \SI{>30}{\minute}, rinsed with \gls{dih2o} copiously and soaked in \gls{h2so4} for \SI{>30}{\minute} as well. Then, the samples were rinsed again in \acrlong{dih2o}. To form silanol groups on the activated surface, the surfaces were finally immersed in \SI{>90}{\degreeCelsius} heated (SuperNuova+, Thermo Scientific) \gls{dih2o}  for at least \SI{2}{\hour}.
\subsubsection{Piranha Activation}
In this method, activation was carried out in a 1:5 (v/v) piranha solution at \SI{70}{\degreeCelsius} for \SI{30}{\minute}. After treatment, the samples were rinsed carefully with \gls{dih2o} three times.
\subsubsection{Hydrofluoric Acid Activation}

\subsection{Chemical Surface Functionalization}
Chemically activated surfaces were now coupled with \gls{aptes} covalently. Therefore an aqueous silane solution was prepared from \gls{etoh} with volume fractions of \SI{5}{\percent} \gls{dih2o}, \SI{0.5}{\percent} aqueous \gls{acoh} (pH 4.5) and \SI{1}{\percent} \gls{aptes} in this order. The samples were soaked immediately after their activation in the silane solution. The reaction was carried out for \SIrange{2}{4}{\hour} at \SI{>40}{\degreeCelsius}. At finish, all specimens were rinsed three times or sonicated for \SI{5}{\minute} in absolute \gls{etoh}.\\
Then, the amine terminated surface modification was enhanced by a carbodiimide conjugation with \gls{paa}. As above, a reaction consisting of \SI{0.5}{\molar} \gls{mes} buffer with \SI{1}{\milli\gram\per\milli\liter}, \SI{6}{\milli\molar} \gls{edc} and  \SI{3}{\milli\molar} \gls{nhs} was activated for \SI{15}{\minute} on a magnetic stirrer. Subsequently, the prepared samples were immersed in the solution for \SI{1}{\hour} on a rotation shaker (VWR). As final cleaning, the slides were rinsed or sonicated for \SI{5}{\minute} in \gls{dih2o} and stored in fresh \gls{dih2o} at \SI{4}{\degreeCelsius} up to \SI{14}{\day} upon further use.\cite{lib:chem:Anti-EpCAM-PAA}
\subsection{Surface Bioconjugation}

\subsection{Particle Functionalization}





