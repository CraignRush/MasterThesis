%%%%%%%%%%%%%%%%%%%%%%%%%%%%%%%%%%%%%%%%%%%%%%%%%%%%%%%%%%%%%%%%%%%%%%%%%%%%%%%%
% EINSTELLUNGEN
%%%%%%%%%%%%%%%%%%%%%%%%%%%%%%%%%%%%%%%%%%%%%%%%%%%%%%%%%%%%%%%%%%%%%%%%%%%%%%%%

\KOMAoptions{parskip=full}

% Seitenränder:

\newcommand{\SeitenrandOben}{30mm}
\newcommand{\SeitenrandRechts}{30mm}
\newcommand{\SeitenrandLinks}{40mm}
\newcommand{\SeitenrandUnten}{30mm}
\newcommand{\FusszeileHoehe}{11.7mm}

\usepackage[a4paper,
    head=0pt,
    top=\SeitenrandOben,
    bottom=\SeitenrandUnten,
    inner=\SeitenrandLinks,
    outer=\SeitenrandRechts
]{geometry}

% Deckblatt:
\newcommand{\UniversitaetLogoBreite}{19mm}
\newcommand{\UniversitaetLogoHoehe}{1cm}

\newcommand{\FotoStudentBreite}{45mm}
\newcommand{\FotoStudentHoehe}{60mm}

\textblockorigin{\SeitenrandLinks}{\SeitenrandOben} 

\setlength{\parindent}{0pt}
%\setlength{\baselineskip}{32pt}
\setlength{\parskip}{\baselineskip}
\TabPositions{4cm}


% Fußzeilen:

\setlength{\footheight}{\FusszeileHoehe}
\clearpairofpagestyles
\ofoot*{\pagemark\vfill}
\setkomafont{pageheadfoot}{\fontsize{9pt}{13pt}\normalfont}
\setkomafont{pagefoot}{\bfseries}
\setkomafont{pagenumber}{\normalfont}
\pagestyle{scrheadings}


% Fußnoten:

%\KOMAoptions{%
%    footnotes=multiple % mehrere Fußnoten werden durch Zeichen getrennt
%}
%\setfootnoterule[.6pt]{5.08cm}
\renewcommand{\footnoterule}{\hrule width 5.08cm height .6pt \vspace*{3.9mm}}
%\setlength{\footnotesep}{5mm}
\deffootnote{2mm}{2mm}{%
    \makebox[2mm][l]{\textsuperscript{\thefootnotemark}}%
}
\setkomafont{footnoterule}{\fontsize{9pt}{20pt}\selectfont}




% Überschriften:

\KOMAoptions{%
    open=any, % keine Festlegung auf linke oder rechte Seite
    numbers=noendperiod, % kein automatischer Punkt nach Gliederungsnummer
    headings=small
}

\makeatletter
\g@addto@macro{\@afterheading}{\vspace{-\parskip}} % \parskip nach Gliederungsbefehlen entfernen
\renewcommand*{\chapterheadstartvskip}{\vspace{\@tempskipa}\vspace{-3pt}} % Korrektur für Abstand über Kapitelüberschriften
\makeatother

\setkomafont{disposition}{\normalfont\sffamily}

\setkomafont{chapter}{\normalfont\fontsize{19pt}{22pt}\selectfont}
\RedeclareSectionCommand[%
  beforeskip=0pt,
  afterskip=29pt
]{chapter}
\renewcommand*{\chapterformat}{\thechapter.\enskip} % Immer Punkt nach Kapitelnummer

\setkomafont{section}{\fontsize{15pt}{17pt}\selectfont}
\RedeclareSectionCommand[%
  beforeskip=0pt,
  afterskip=24.1pt
]{section}
\renewcommand*{\sectionformat}{\makebox[13mm][l]{\thesection.\enskip}} % Feste Breite für Abschnittsnummer und immer Punkt danach

\setkomafont{subsection}{\bfseries\fontsize{12pt}{13pt}\selectfont}
\RedeclareSectionCommand[%
  beforeskip=0pt,
  afterskip=16pt
]{subsection}
\renewcommand*{\subsectionformat}{\makebox[13mm][l]{\thesubsection.\enskip}} % Feste Breite für Unterabschnittsnummer und immer Punkt danach
\setkomafont{subsubsection}{\bfseries\fontsize{11pt}{12pt}\selectfont}
\RedeclareSectionCommand[%
beforeskip=0pt,
afterskip=16pt
]{subsubsection}

% Listen:

\setlist{%
    labelsep=0mm,
    itemindent=0pt,
    labelindent=0pt,
    align=left,
    parsep=1.5ex
}
\setlist[itemize]{%
    leftmargin=5mm,
    labelwidth=4.9mm
}
\setlist[itemize,1]{%
    before={\vspace{0.25ex}},
    label={\raisebox{.35ex}{\smaller[2]\textbullet}},
    after={\vspace{-\parsep}\vspace{-.25ex}}
}
\setlist[itemize,2]{%
    label={\raisebox{.35ex}{\rule{.58ex}{.58ex}}}
}
\setlist[enumerate]{%
    leftmargin=10mm,
    labelwidth=9.9mm
}
\setlist[enumerate,2]{%
    label={\alph*.}
}

\setlist[description]{%
%    labelindent=!,
    leftmargin=1em,
    labelwidth=!,
    parsep=0mm,
    partopsep=0mm,
    labelsep=1em,
}


% Verzeichnisse:

\KOMAoptions{%
    %toc=flat, % keine Einrückungen im Inhaltsverzeichnis
    toc=chapterentrydotfill, % Punkte bis zur Seitennummer bei Kapiteln
    listof=entryprefix, % Präfix für Einträge in Abbildungs- und Tabellenverzeichnis
   	listof=nochaptergap, % Kein Abstand für Kapiteleinträge in extra Verzeichnissen
}

\makeatletter
\setkomafont{chapterentry}{\normalsize\bfseries}  % Chapter soll fett dargestellt werden
\renewcommand{\@dotsep}{.3} % Abstand der Füllpunkte

% "chapteratlist" für Inhaltsverzeichnis auswerten:
\renewcommand*{\addchaptertocentry}[2]{%
  \iftocfeature{toc}{chapteratlist}{}{%
    \addtocontents{toc}{\protect\vspace{-10pt}}% extra Abstand vor Kapitelüberschriften in Inhaltsverzeichnis entfernen
  }%
  % Originaldefinition aus scrbook.cls:
  \addtocentrydefault{chapter}{#1}{#2}%
  \if@chaptertolists
    \doforeachtocfile{%
      \iftocfeature{\@currext}{chapteratlist}{%
        \addxcontentsline{\@currext}{chapteratlist}[{#1}]{#2}%
      }{}%
    }%
    \@ifundefined{float@addtolists}{}{\scr@float@addtolists@warning}%
  \fi%
}

\makeatother

\AfterTOCHead[toc]{\protect\vspace{.8ex}} % Abstand zwischen Überschrift und Inhaltsverzeichnis
\setuptoc{toc}{noparskipfake} % Angleichung der Abstände nach Inhaltsverzeichnisüberschrift an andere Überschriften
\unsettoc{toc}{chapteratlist} % kein Abstand vor Kapiteleinträgen im Inhaltsverzeichnis, funktioniert nur durch obige Redefinition von \addchaptertocentry

% -- Abbildungs- und Tabellenverzeichnis:

\AfterTOCHead[lof]{\protect\vspace{-.1ex}\doublespacing} % Abstand zwischen Überschrift und Abbildungsverzeichnis, doppelter Zeilenabstand
\setuptoc{lof}{noparskipfake} % Angleichung der Abstände nach Abbildungsverzeichnisüberschrift an andere Überschriften

\AfterTOCHead[lot]{\protect\vspace{-.1ex}\doublespacing} % Abstand zwischen Überschrift und Tabellenverzeichnis, doppelter Zeilenabstand
\setuptoc{lot}{noparskipfake} % Angleichung der Abstände nach Tabellenverzeichnisüberschrift an andere Überschriften



% Tabellen:
\renewcommand{\arraystretch}{1.8} % Skalierung der Tabellen
\newcolumntype{M}{X<{\vspace{2pt}}} % Spaltentyp mit Abstand rechts


% Glossare & Abkürzungsverzeichnis:
\makeglossaries
\setacronymstyle{long-short}
%\glsaddall

\makeatletter
\newlength{\@glsdotsep}
\setlength{\@glsdotsep}{\@dotsep em}
\newcommand*{\glsdotfill}{\leavevmode \cleaders \hb@xt@ \@glsdotsep{\hss .\hss }\hfill \kern \z@}
\makeatother

\newglossarystyle{WissenschaftlicheArbeiten}{%
\setglossarystyle{index}%

  \renewcommand*{\glossaryheader}{\vspace{.75em}}%
  \renewcommand*{\glstreenamefmt}[1]{##1}%
  \renewcommand*{\glossentry}[2]{%
     \item\glsentryitem{##1}\glstreenamefmt{\glstarget{##1}{\glossentryname{##1}}}%
     \ifglshassymbol{##1}{\space(\glossentrysymbol{##1})}{}%
     \space-\space\glossentrydesc{##1}\glsdotfill\glspostdescription\space ##2%
  }%
  \renewcommand*{\glsgroupheading}[1]{%
    \item\glstreenamefmt{\textbf{\fontsize{14}{17}\selectfont\enskip\glsgetgrouptitle{##1}}}\vspace{.3em}}%
}

\setglossarystyle{WissenschaftlicheArbeiten}




%%%%%%%%%%%%%%%%%%%%%%%%%%%%%%%%%%%%%%%%%%%%%%%%%%%%%%%%%%%%%%%%%
%%%%%% O W N   C O M M A N D S %%%%%%%%%%%%%%%%%%%%%%%%%%%%%%%%%%

%%% Equation for reactions
%\chemsetup{
%formula = mhchem,
%reactions/tag-open    = {(},
%reactions/tag-close   = {)},
%}
%\crefname{reaction}{reaction}{reactions}
%\creflabelformat{reaction}{#2\{#1\}#3}  

\usepackage[textsize=tiny]{todonotes} % get todonotes
%\setlength{\marginparwidth}{25mm}

% Equation numbering style
\makeatletter
\renewcommand\tagform@[1]{\maketag@@@{\ignorespaces#1\unskip\@@italiccorr}}
\makeatother

\renewcommand{\theequation}{\arabic{section}.\arabic{equation}}
\creflabelformat{equation}{#2\textup{#1}#3}

% Set figure Enumeration macro
% -- Abbildungen:

\newcommand\capption[2]{\caption{\textbf{#1}\newline#2}}


\renewcommand{\thesubfigure}{\alph{subfigure}}
\renewcommand{\thesubtable}{\alph{subtable}}

% Beschriftungen:
\DeclareCaptionFormat{WissenschaftlicheArbeiten}{\fontsize{8pt}{10pt}\selectfont#1 #2#3\par}
\DeclareCaptionLabelFormat{WissenschaftlicheArbeiten}{\bfseries\fontsize{8pt}{10pt}\selectfont#1 #2:      }
\DeclareCaptionLabelFormat{OwnBold}{\bfseries\fontsize{14pt}{16pt}\selectfont#1 #2:}

% -- Tabellen:
\captionsetup[table]{%
	format=WissenschaftlicheArbeiten,
	labelformat=WissenschaftlicheArbeiten,
	labelsep=none,
	singlelinecheck=off,
	justification=raggedright,
	skip=3pt,
	tablewithin=none
}

% -- Abbildungen:
\setlength{\intextsep}{\baselineskip}
\setlength{\textfloatsep}{\baselineskip}
\setlength{\floatsep}{\baselineskip}

\captionsetup[figure]{%
	format=WissenschaftlicheArbeiten,
	labelformat=WissenschaftlicheArbeiten,
	labelsep=none,
	singlelinecheck=off,
	justification=raggedright,
	%skip=3pt,%6.6mm
	belowskip=0pt,
	figurewithin=none%
}

\captionsetup[subtable]{%
	position=top, 
	labelfont={large,bf},
	singlelinecheck=off,
	justification=raggedright,
	labelformat=simple,
	labelsep=none
}

\Crefname{equation}{Eq.}{Eqs.}
\Crefname{figure}{Fig.}{Figs.}
\Crefname{tabular}{Tab.}{Tabs.}
\Crefname{section}{Sec.}{Secs.}

\newcommand{\subfigimg}[3][,]{%
	\setbox1=\hbox{\includegraphics[#1]{#3}}% Store image in box
	\leavevmode\rlap{\usebox1}% Print image
	%\rlap{\hspace*{0pt}\raisebox{\dimexpr\ht1+0.25\baselineskip}{\fontsize{8pt}{10pt}\bfseries\selectfont\thesubfigure}}% Print label #2
	\rlap{\hspace*{0pt}\raisebox{\dimexpr\ht1+0.25\baselineskip}{\large\bfseries #2}}%\selectfont#2
	\phantom{\usebox1}% Insert appropriate spcing	
}

% Get figure colors as text bricks
\newlength{\tikzLineHeight}
\setlength{\tikzLineHeight}{-0.25em}
\newlength{\tikzLineLength}
\setlength{\tikzLineLength}{5mm}
\definecolor{MLblue}{rgb}{0, 0.4470, 0.7410}
\definecolor{MLorange}{rgb}{0.8500, 0.3250, 0.0980}
\definecolor{Oorange}{HTML}{E69F00}
\definecolor{OlightBlue}{HTML}{56B4E9}
\definecolor{Ogreen}{HTML}{009E73}
\definecolor{OdarkBlue}{HTML}{0072B2}
\definecolor{Oyellow}{HTML}{F0E442}
\definecolor{Ored}{HTML}{D55E00}

\DeclareRobustCommand{\orangedash}{
	\tikz[baseline=\tikzLineHeight]{	
		\draw[-,color=Oorange,dashed,line width = 1.5pt](0,0) -- (\tikzLineLength,0);
	}	
}

\DeclareRobustCommand{\orangeline}{
	\tikz[baseline=-0.3\baselineskip]{
		\draw[-,color=Oorange,solid,line width = 1.5pt](0,0) -- (\tikzLineLength,0);
	}
}

\DeclareRobustCommand{\blueline}{
	\tikz[baseline=\tikzLineHeight]{	
		\draw[-,color=OlightBlue,solid,line width = 1.5pt](0,0) -- (\tikzLineLength,0);
	}
}

\DeclareRobustCommand{\bluedash}{
	\tikz[baseline=\tikzLineHeight]{	
		\draw[-,color=OlightBlue,dashed,line width = 1.5pt](0,0) -- (\tikzLineLength,0);
	}
}

\DeclareRobustCommand{\yellowline}{
	\tikz[baseline=\tikzLineHeight]{	
			\draw[-,Oyellow,solid,line width = 1.5pt](0,0) -- (\tikzLineLength,0);
		}
}

\DeclareRobustCommand{\greenline}{
	\tikz[baseline=\tikzLineHeight]{	
		\draw[-,Ogreen,solid,line width = 1.5pt](0,0) -- (\tikzLineLength,0);
	}
}

\DeclareRobustCommand{\blackline}{
	\tikz[baseline=\tikzLineHeight]{	
			\draw[-,black!,solid,line width = 1.5pt](0,0) -- (\tikzLineLength,0);
		}
}

\DeclareRobustCommand{\orangerect}{
	\tikz[baseline=\tikzLineHeight]{
		\draw[fill=Oorange,draw=none](0.375\tikzLineLength,-0.16\tikzLineLength) rectangle (0.625\tikzLineLength,0.16\tikzLineLength);
		\draw[-,Oorange,solid,line width = 1.5pt](0,0) -- (\tikzLineLength,0);
	}			
}

\DeclareRobustCommand{\bluerect}{
\tikz[baseline=\tikzLineHeight]{
	\draw[fill=OlightBlue,draw=none](0.375\tikzLineLength,-0.16\tikzLineLength) rectangle (0.625\tikzLineLength,0.16\tikzLineLength);
	\draw[-,OlightBlue,solid,line width = 1.5pt](0,0) -- (\tikzLineLength,0);
	}			
}
\DeclareRobustCommand{\yellowrect}{
\tikz[baseline=\tikzLineHeight]{
	\draw[fill=Oyellow,draw=none](0.375\tikzLineLength,-0.16\tikzLineLength) rectangle (0.625\tikzLineLength,0.16\tikzLineLength);
	\draw[-,Oyellow,solid,line width = 1.5pt](0,0) -- (\tikzLineLength,0);
	}			
}
\DeclareRobustCommand{\greenrect}{
\tikz[baseline=\tikzLineHeight]{
	\draw[fill=Ogreen,draw=none](0.375\tikzLineLength,-0.16\tikzLineLength) rectangle (0.625\tikzLineLength,0.16\tikzLineLength);
	\draw[-,Ogreen,solid,line width = 1.5pt](0,0) -- (\tikzLineLength,0);
	}			
}
\DeclareRobustCommand{\blackrect}{
\tikz[baseline=\tikzLineHeight]{
	\draw[fill=black,draw=none](0.375\tikzLineLength,-0.16\tikzLineLength) rectangle (0.625\tikzLineLength,0.16\tikzLineLength);
	\draw[-,black,solid,line width = 1.5pt](0,0) -- (\tikzLineLength,0);
	}			
}


\DeclareRobustCommand{\blueCircle}{\tikz{\draw[black, fill=OlightBlue] (0,0) circle (.5ex);}}

\DeclareRobustCommand{\darkBlueCircle}{\tikz{\draw[black, fill=OdarkBlue] (0,0) circle (.5ex);}}

\DeclareRobustCommand{\orangeCircle}{\tikz{\draw[black, fill=Oorange] (0,0) circle (.5ex);}}