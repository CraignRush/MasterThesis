
\chapter{Abstract}
Biosensors employ usually optical \cite{lit:bio:BioconjugateTechniques}, electrochemical \cite{lit:fluidic:BindingPhysicsSurfaces} or magnetic \cite{lit:thes:reisbeck, lit:bio:MRCyte2016, lit:bio:NanoCytometer} transducers to detect biomarkers.  Optical  biosensors  have  manifested  high  sensitivity.\cite{lit:Shapiro}  For example, one of the  most  sensitive  optical  biosensors reached a detection limit of \SI{1}{cfu \per\milli\liter} \textit{E. coli} within about half an hour,  combining  a  flow  cytometer  and  fluorescent  \glspl{mnp}.\cite{lit:flowCytometer} \\
Magnetic flow cytometry targets shortcomings of optical systems under the trade-off, that until now lower throughput and less colors are possible to measure.\cite{lit:thes:reisbeck} Especially in optically dense samples, such as human body fluids and blood, magnetic flow cytometry can show its superior capabilities because of a negligible magnetic background in the biological samples. On the one side, this enables rapid point-of-care tests since the \gls{gmr} sensor costs less than \EUR{20} in comparison to optical spectrometers or cameras.\cite{lit:fluidic:HighFlowGMR,lit:bio:aflatoxinMNP,lit:bio:POC_CD64} On the other side, the sharp sensitivity and electronic speed of these sensors makes precise single cell measurements possible.\cite{lit:bio:MRCyte2016,lit:fluidic:GMR_Quantification, lit:paperHelou} When magnetically labeled cells are rolling over such a sensor, information about a cell's size, morphology or biomarker density can be extracted from a single signal pattern.\cite{lit:thes:michaelBauer} 

%After determining a detection limit for the velocity decrease, we aimed to achieve complete
%immobilization of analytes on the sensor surface. We quantitatively tested the effectiveness
%using optical microscopy. We did not find immobilization based on covalent bindings and
%abandoned this method. However, we successfully achieved immobilization based on physical
%adsorption on PDSM/glass with reference particles and on SiN with reference particles,
%cells from cell culture and cells from peripheral blood. We further developed a protocol for
%analyte deceleration instead of complete immobilization by increasing the flow rate, or decreasing
%the number of interaction sites. Based on qualitative optical evaluation of the analyte
%velocity, we found significant differences of analyte velocity between functionalized and
%non-functionalized surfaces. However, optical read-out limits our analysis and we cannot
%differentiate very low velocities from complete immobilization. In the future, the protocol
%should be evaluated using a magnetic read-out.
%We presented functionalization of the sensor surface as a method to add velocity as an additional
%measurement dimension to magnetic flow cytometry. We verified complete immobilization
%of different analytes, but final confirmation of the decrease in velocity is still pending.
In this thesis, the interaction of cells with a bio-functionalized surface under laminar flow conditions will be simulated from a hydrodynamic and magnetic point of view. Then, a reference system for the variation of surface receptor density will be established and subsequently evaluated. Ultimately, an affinity-based concentration assay will be presented which reveals promising results for the magnetic measurement of biomarker concentrations or single cell surface proteins in the future.
