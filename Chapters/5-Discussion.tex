\chapter{Discussion}
During the course of this thesis, motions of spheroids - such as beads or cells - over the \gls{gmr}-sensor have been modeled from a physics and signal processing perspective. In a magnetic field simulation it has been found that beads generate a well defined signal shape if the \glspl{mnp} are distributed homogeneously over their respective surface. Thereby, the size of each \gls{mnp} matters strongly. It is a strong trade-off between carrying enough magnetic momentum to get a sufficient amplitude in dipole superposition and at the same time not causing inhomogeneities in the magnetic signal. Nevertheless, a broad range from \SIrange{100}{1000}{\nano\meter} diameter of the labels showed a linear magnetization behavior above \SI{30}{\percent} of total coverage.\\
In the force-equilibrium simulation, a model for the bead rolling motion on the bottom of a functionalized microchannel was built on basic calculus estimations. Apparently, the dominant rolling forces on a microbead act in the \si{\nano\newton} range, which leads to a necessary interaction from a single to thousand proteins according to their specific release force per bond. This could be subject to further refinements of the MATLAB framework with an implementation of differential equations for binding kinetics \cite{lit:bio:BindingModeling} or the rolling motion to gain a more crisp representation of the reality. 

Those computational findings were adopted in a method to create different surface densities of biotin or antibodies on \SI{8}{\micro\meter}-beads by carbodiimide chemistry. On the one side, the magnetic momentum of beads could now be varied through a saturation with streptavidin-\gls{mnp} conjugates and measured in the magnetic flow cytometer. On the other side, this allowed for a magnetic bead capture assay where an unspecifically functionalized channel prevented the rolling of beads coated with a matching ligand. Additionally, for a more robust force measurement in future experiments, methods for the covalent surface modification of glass, \gls{pdms} and \gls{sin} have been established. These modifications were based on different wet-chemistry oxidation techniques, with subsequent silane and carbodiimide chemistry. After bio-functionalization, they were then analyzed by fluorescence microscopy or an optical bead capture assay and showed a similar surface quality with respect to physisorption.\\
However, the covalent nature - or more precisely the non-physisorption -  of the surface modification as well as their projected advantages in stability and definedness could not be demonstrated during this thesis. Also, the bead rolling assay on each substrate had never been evaluated magnetically but only optically due to the lack of sensors on the respective substrates. Thus, a quantitative comparison between both measurements bears much room for misassumptions. In addition, bio-functionalized surfaces have never been brought into contact with any physiological substances with blood or immune cells which are very likely to disturb the operation.
Due to the underlying biochemical processes in mostly all experiments, some effects remain unexplained: First, fluorescent labeling of biotinylated beads with streptavidin-atto488 yielded robust and reliable results in the bead characterization in a optical flow cytometer. In contrast, a similar labeling process with a recombinant secondary-antibody (REA746-PE, REAfinity, Miltenyi Biotech, Bergisch Gladbach, Germany) showed inexplainable results. This could very likely be related to a sterically blocked binding domain but remains a question-to-be-solved.\\
Yet, another effect is related to \gls{mnp} coverage of bio-functionalized polystyrene microbeads.

 Notwithstanding, the validated protocols present a comprehensive groundwork for further experiments regarding affinity measurements, 
 

Start by briefly summarizing your major findings, but without repeating exact data from the Results. This makes your novel information clear to peer reviewers and, later on, readers. It also forces you to decide which findings you should focus on in the Discussion.
Thereafter, discuss possible underlying mechanisms. Why did you get these results, what is happening? Mechanisms, particularly molecular mechanisms, have very high “impact” in the natural sciences.

Next, compare your findings to those of other relevant publications and attempt to explain any discrepancies. If your findings disagree with those of others in the area, compare their publication to your own manuscript in minute detail, looking for any differences (especially in methodology) that might explain the discrepancy.

Consider the possible limitations of your own study—paradoxically, most reviewers consider an awareness and openness about potential weaknesses as a strength. However, do not forget to emphasize strengths as well.

Discuss the possible consequences of your observations and/or future investigations required or motivated. Be as concrete as possible about future perspectives. As in the abstract, writing the equivalent of “more research is needed” is meaningless—more research is always needed. Describe the hypotheses, questions or mechanisms that need to be investigated and/or methods that should be applied, concretely and concisely. And, of course, if your research findings have potential practical implications, discuss these in some detail as well.

Finally, state your conclusions—have you supported or rejected the hypothesis you posed, or obtained an answer to your research question?





eriving Navier-Stokes equation by the Cauchy momentum equation is complex and harbors several sources of error. First, an incompressible Newtonian fluid as well as channel boundary is assumed. The used water suspensions are approximated with negligible compressibility, which is not true for the real case. Also, for blood or other shear-thinning fluids these deviations are prone for high errors.  

Then, the divergence relation of the respective viscous stress (eq. \ref{eq:divergence_Stresstensor}) does not hold for non-uniform viscosity $\eta$.

For later studies in a matlab model, the flow velocity and shear stress computations were carried out with the error sources considered. 




Second, the channel height varies in reality as a result of fabrication inaccuracies. 

modification of nh2 with paa and protein like cooh

Contact angle for silanization of surface methods more useful --> should be 1st approach for characterization

Anti-Biotin-PE working?
BNF-Dextran-Streptavidin unspecific binding?
electrostatic surface interaction
evidence covalent binding?

gas bubbles, adsprotion decrease, tubing exchange
activity of protein

bead rolling with biotin-cooh, biotin-nh2, plain-cooh, plain-nh2 --> velocity correction factor

concentration of beads in differential setup too high, maybe better results with lower.