\chapter{Outlook}
The validated protocols in this thesis for bead and surface bio-functionalization present a comprehensive groundwork for further experiments regarding affinity measurements, magnetic bead or cell trapping, and exosome assays. Also, a computational and practical model for the cell receptor density has been characterized extensively and could be used for a further study of the magnetic flow cytometer or even as internal standard in assay development. 

The whole magnetic flow cytometer, nevertheless, offers some elements for potential improvements: \\
First, the chip design could be tailored more to the respective sensor application in order to achieve beneficial flow patterns, less bead losses, or the filtering of bubbles.\\
Second, the photo-lithographic sensor design could be improved by its sensitivity for each beads and cells. The distance between \gls{gmr}-bridges could be increased for an enhanced resolution of time-of-flight and peak overlap with the compromise of a further limited, measurable concentration. Also, whereas cells require high sensitivity and low noise along with magnetic focusing, magnetic beads could be measured with a channel spanning \gls{gmr}-setup to avoid the additional complexity of the magnetophoretic nickel-iron structures. Especially in the light of highly sensitive single-molecule or affinity measurements, the \acrlong{mr} could be leveraged by exploiting the tunnel-magneto-resistance instead of the giant-magneto-resistance in a new sensor layout.\cite{lit:GMRTMR}\\
Third, from a time- and work-efficiency perspective, it could be profitable to develop a (real-time, automatized) data analysis software by a model-based wavelet correlation or compressive sensing. Several algorithms have already been developed \cite{lit:thes:michaelBauer}, however, experimental verification and overall back-end integration is required.\\
Lastly, the bio-functionalization could be benchmarked for its attachment mechanism from a chemistry and physical standpoint by field-typical methods such as dynamic-light scattering, activity assays, electron/atomic-force microscopy, or spectroscopy. This would provide deeper insights into surface composition and hence the covalent to physisorption efficiency or model parameters for the bead rolling.\\
Experimentally, several pathways can now be explored. The bead rolling and its subsequent motion inside a plain channel could be related its electrostatic surface interaction by testing for example \gls{carboxyl}, \gls{amine}, and biotinylated particles as well as their intermediates. Also a more precise velocity correction factor could be formulated.\\
Also, the differential counting setup should not be omitted prematurely. By removing the tubing interconnection, for instance by a drilled via, many losses could be lowered. Moreover, the utilized high bead concentrations did probably not reflect the ultimate capabilities during a time-of-flight measurement.

Despite the elaborated improvement options, the magnetic flow cytometry system starts to open a new field in clinical diagnostics and point-of-care testing, where a measurement is completely independent from the optical background of blood samples. With the proper integration level, tests for immune cell counts or inflammation intensities\footnote{measured by the cell marker density and the subsequent adhesiveness to an antibody-functionalized surface} could be performed bedside in minutes and effectively start saving lives in the near future. 


