%%%%%%%%%%%%%%%%%%%%%%%%%%%%%%%%%%%%%%%%%%%%%%%%%%%%%%%%%%%%%%%%%%%%%%%%%%%%%%%%
% EINSTELLUNGEN
%%%%%%%%%%%%%%%%%%%%%%%%%%%%%%%%%%%%%%%%%%%%%%%%%%%%%%%%%%%%%%%%%%%%%%%%%%%%%%%%

\KOMAoptions{parskip=full}

% Seitenränder:

\newcommand{\SeitenrandOben}{30mm}
\newcommand{\SeitenrandRechts}{30mm}
\newcommand{\SeitenrandLinks}{40mm}
\newcommand{\SeitenrandUnten}{30mm}
\newcommand{\FusszeileHoehe}{11.7mm}

\usepackage[a4paper,
    head=0pt,
    top=\SeitenrandOben,
    bottom=\SeitenrandUnten,
    inner=\SeitenrandLinks,
    outer=\SeitenrandRechts
]{geometry}

% Deckblatt:
\newcommand{\UniversitaetLogoBreite}{19mm}
\newcommand{\UniversitaetLogoHoehe}{1cm}

\newcommand{\FotoStudentBreite}{45mm}
\newcommand{\FotoStudentHoehe}{60mm}

\textblockorigin{\SeitenrandLinks}{\SeitenrandOben} 

\setlength{\parindent}{0pt}
%\setlength{\baselineskip}{32pt}
\setlength{\parskip}{\baselineskip}
\TabPositions{4cm}


% Fußzeilen:

\setlength{\footheight}{\FusszeileHoehe}
\clearscrheadfoot
\ofoot*{\pagemark\vfill}
\setkomafont{pageheadfoot}{\fontsize{9pt}{13pt}\normalfont}
\setkomafont{pagefoot}{\bfseries}
\setkomafont{pagenumber}{\normalfont}
\pagestyle{scrheadings}


% Fußnoten:

%\KOMAoptions{%
%    footnotes=multiple % mehrere Fußnoten werden durch Zeichen getrennt
%}
%\setfootnoterule[.6pt]{5.08cm}
\renewcommand{\footnoterule}{\hrule width 5.08cm height .6pt \vspace*{3.9mm}}
%\setlength{\footnotesep}{5mm}
\deffootnote{2mm}{2mm}{%
    \makebox[2mm][l]{\textsuperscript{\thefootnotemark}}%
}
\setkomafont{footnoterule}{\fontsize{9pt}{20pt}\selectfont}


% Überschriften:

\KOMAoptions{%
    open=any, % keine Festlegung auf linke oder rechte Seite
    numbers=noendperiod, % kein automatischer Punkt nach Gliederungsnummer
    headings=small
}

\makeatletter
\g@addto@macro{\@afterheading}{\vspace{-\parskip}} % \parskip nach Gliederungsbefehlen entfernen
\renewcommand*{\chapterheadstartvskip}{\vspace{\@tempskipa}\vspace{-3pt}} % Korrektur für Abstand über Kapitelüberschriften
\makeatother

\setkomafont{disposition}{\normalfont\sffamily}

\setkomafont{chapter}{\normalfont\fontsize{19pt}{22pt}\selectfont}
\RedeclareSectionCommand[%
  beforeskip=0pt,
  afterskip=29pt
]{chapter}
\renewcommand*{\chapterformat}{\thechapter.\enskip} % Immer Punkt nach Kapitelnummer

\setkomafont{section}{\fontsize{15pt}{17pt}\selectfont}
\RedeclareSectionCommand[%
  beforeskip=0pt,
  afterskip=24.1pt
]{section}
\renewcommand*{\sectionformat}{\makebox[13mm][l]{\thesection.\enskip}} % Feste Breite für Abschnittsnummer und immer Punkt danach

\setkomafont{subsection}{\bfseries\fontsize{12pt}{13pt}\selectfont}
\RedeclareSectionCommand[%
  beforeskip=0pt,
  afterskip=1pt
]{subsection}
\renewcommand*{\subsectionformat}{\makebox[13mm][l]{\thesubsection.\enskip}} % Feste Breite für Unterabschnittsnummer und immer Punkt danach
\setkomafont{subsubsection}{\bfseries\fontsize{11pt}{12pt}\selectfont}
\RedeclareSectionCommand[%
beforeskip=0pt,
afterskip=1pt
]{subsubsection}

% Listen:

\setlist{%
    labelsep=0mm,
    itemindent=0pt,
    labelindent=0pt,
    align=left,
    parsep=1.5ex
}
\setlist[itemize]{%
    leftmargin=5mm,
    labelwidth=4.9mm
}
\setlist[itemize,1]{%
    before={\vspace{0.25ex}},
    label={\raisebox{.35ex}{\smaller[2]\textbullet}},
    after={\vspace{-\parsep}\vspace{-.25ex}}
}
\setlist[itemize,2]{%
    label={\raisebox{.35ex}{\rule{.58ex}{.58ex}}}
}
\setlist[enumerate]{%
    leftmargin=10mm,
    labelwidth=9.9mm
}
\setlist[enumerate,2]{%
    label={\alph*.}
}

\setlist[description]{%
%    labelindent=!,
    leftmargin=1em,
    labelwidth=!,
    parsep=0mm,
    partopsep=0mm,
    labelsep=1em,
}


% Verzeichnisse:

\KOMAoptions{%
    %toc=flat, % keine Einrückungen im Inhaltsverzeichnis
    toc=chapterentrydotfill, % Punkte bis zur Seitennummer bei Kapiteln
    listof=entryprefix, % Präfix für Einträge in Abbildungs- und Tabellenverzeichnis
   	listof=nochaptergap, % Kein Abstand für Kapiteleinträge in extra Verzeichnissen
}

\makeatletter
\setkomafont{chapterentry}{\normalsize\bfseries}  % Chapter soll fett dargestellt werden
\renewcommand{\@dotsep}{.3} % Abstand der Füllpunkte

% "chapteratlist" für Inhaltsverzeichnis auswerten:
\renewcommand*{\addchaptertocentry}[2]{%
  \iftocfeature{toc}{chapteratlist}{}{%
    \addtocontents{toc}{\protect\vspace{-10pt}}% extra Abstand vor Kapitelüberschriften in Inhaltsverzeichnis entfernen
  }%
  % Originaldefinition aus scrbook.cls:
  \addtocentrydefault{chapter}{#1}{#2}%
  \if@chaptertolists
    \doforeachtocfile{%
      \iftocfeature{\@currext}{chapteratlist}{%
        \addxcontentsline{\@currext}{chapteratlist}[{#1}]{#2}%
      }{}%
    }%
    \@ifundefined{float@addtolists}{}{\scr@float@addtolists@warning}%
  \fi%
}

\makeatother

\AfterTOCHead[toc]{\protect\vspace{.8ex}} % Abstand zwischen Überschrift und Inhaltsverzeichnis
\setuptoc{toc}{noparskipfake} % Angleichung der Abstände nach Inhaltsverzeichnisüberschrift an andere Überschriften
\unsettoc{toc}{chapteratlist} % kein Abstand vor Kapiteleinträgen im Inhaltsverzeichnis, funktioniert nur durch obige Redefinition von \addchaptertocentry

% -- Abbildungs- und Tabellenverzeichnis:

\AfterTOCHead[lof]{\protect\vspace{-.1ex}\doublespacing} % Abstand zwischen Überschrift und Abbildungsverzeichnis, doppelter Zeilenabstand
\setuptoc{lof}{noparskipfake} % Angleichung der Abstände nach Abbildungsverzeichnisüberschrift an andere Überschriften

\AfterTOCHead[lot]{\protect\vspace{-.1ex}\doublespacing} % Abstand zwischen Überschrift und Tabellenverzeichnis, doppelter Zeilenabstand
\setuptoc{lot}{noparskipfake} % Angleichung der Abstände nach Tabellenverzeichnisüberschrift an andere Überschriften

% Beschriftungen:
\DeclareCaptionFormat{WissenschaftlicheArbeiten}{\fontsize{8pt}{10pt}\selectfont#1 #2#3\par}
\DeclareCaptionLabelFormat{WissenschaftlicheArbeiten}{\bfseries\selectfont#1 #2}

% -- Tabellen:
\captionsetup[table]{%
    format=WissenschaftlicheArbeiten,
    labelformat=WissenschaftlicheArbeiten,
    labelsep=none,
    singlelinecheck=off,
    justification=raggedright,
    skip=3pt,
    tablewithin=none
}

% -- Abbildungen:
\captionsetup[figure]{%
    format=WissenschaftlicheArbeiten,
    labelformat=WissenschaftlicheArbeiten,
    labelsep=none,
    singlelinecheck=off,
    justification=raggedright,
    skip=6.6mm,
    figurewithin=none
}


% Tabellen:
\renewcommand{\arraystretch}{1.8} % Skalierung der Tabellen
\newcolumntype{M}{X<{\vspace{4pt}}} % Spaltentyp mit Abstand rechts


% Glossare & Abkürzungsverzeichnis:


\newacronym{gmr}{GMR}{Giant Magneto Resistance}

\newacronym{pdms}{PDMS}{Poly(dimethyl siloxane)}

\newacronym{sin}{SiN}{Silicon Nitride}

\newacronym{fm}{FM}{Ferrimagnetism}

\newacronym{pm}{PM}{Paramagnetism}

\newacronym{afm}{AFM}{Anti-Ferromagnetism}

\newacronym{nfm}{NFM}{non-ferro-magnetic}

\newacronym{spm}{SPM}{Superparamagnetism}

\newacronym{aaf}{AAF}{Artificial Anti-Ferromagnet}

\newacronym[longplural={Magnetic Nanoparticles}]{MNP}{MNP}{Magnetic Nanoparticle}

\newacronym{IPA}{IPA}{Isopropanol}

\newacronym{pcb}{PCB}{Printed Circuit Board}

\newacronym{gui}{GUI}{Graphical User Interface}

\newacronym{fwhm}{FWHM}{Full Width at Half Maximum}

\newacronym{pbs}{PBS}{Phosphate Buffered Saline}

\newacronym{macs}{MACS}{MACS running buffer}

\newacronym{hcl}{HCl}{Hydrochloric Acid}

\newacronym{h2so4}{H$_{2}$SO$_{4}$}{Sulfuric Acid}

\newacronym{sma}{SMA}{Styrene Maleic Anhydride}

\newacronym{uf}{µF}{Microfluidic}

\newacronym{dih2o}{diH$_2$O}{deionized water}

\newacronym{n2}{N$_2$}{Nitrogen Gas}

\newacronym{o2}{O$_2$}{Oxygen Gas}

\newacronym{etoh}{EtOH}{Ethanol}

\newacronym{meoh}{MeOH}{Methanol}

\newacronym{acoh}{AcOH}{Acetic Acid}

\newacronym{piranha}{Piranha}{H$_2$O$_2$:H$_{2}$SO$_{4}$}

\newacronym{hf}{HF}{Hydrofluoric Acid}

\newacronym{h2o2}{H$_2$O$_2$}{Hydrogen Peroxide}

\newacronym{aptes}{APTES}{(3-aminopropyl)triethoxysilane}

\newacronym{paa}{PAA}{Poly(acrylic) Acid}

\newacronym{edc}{EDC}{1-Ethyl-3-(3-dimethylaminopropyl)carbodiimide}

\newacronym{nhs}{NHS}{N-hydroxysuccinimide}

\newacronym{mes}{MES}{2-(N-morpholino)ethanesulfonic Acid}

\newacronym{u}{u}{flow field}

\newacronym{rho}{$\rho$}{density}

\newacronym{eta}{$\eta$}{dynamic viscosity}

\newacronym{bodyforce}{$\sum_{i}\mathbf{f}_i$}{body forces}

\newacronym{tau}{$\boldsymbol{\tau}$}{surface stress tensor}


\makeglossaries
\setacronymstyle{short-long}


\makeatletter
\newlength{\@glsdotsep}
\setlength{\@glsdotsep}{\@dotsep em}
\newcommand*{\glsdotfill}{\leavevmode \cleaders \hb@xt@ \@glsdotsep{\hss .\hss }\hfill \kern \z@}
\makeatother

\newglossarystyle{WissenschaftlicheArbeiten}{%
  \setglossarystyle{index}%

  \renewcommand*{\glossaryheader}{\vspace{.75em}}%
  \renewcommand*{\glstreenamefmt}[1]{##1}%
  \renewcommand*{\glossentry}[2]{%
     \item\glsentryitem{##1}\glstreenamefmt{\glstarget{##1}{\glossentryname{##1}}}%
     \ifglshassymbol{##1}{\space(\glossentrysymbol{##1})}{}%
     \space-\space\glossentrydesc{##1}\glsdotfill\glspostdescription\space ##2%
  }%
  \renewcommand*{\glsgroupheading}[1]{%
    \item\glstreenamefmt{\textbf{\fontsize{14}{17}\selectfont\enskip\glsgetgrouptitle{##1}}}\vspace{.3em}}%
}

\setglossarystyle{WissenschaftlicheArbeiten}
