\documentclass[%
    fontsize=11pt, % Schriftgröße
    twoside=on % kein einseitiges Layout
]{scrbook} % Dokumentenklasse: KOMA-Script Book
\usepackage{scrlayer-scrpage} % Anpassbare Kopf- und Fußzeilen

\usepackage[utf8]{inputenc} % Textkodierung: UTF-8
\usepackage[T1]{fontenc} % Zeichensatzkodierung

\usepackage[ngerman, english]{babel} % Deutsche Lokalisierung
\usepackage{graphicx} % Grafiken
\usepackage{wrapfig} % Text an Bilder und Tabellen vorbeifließen lassen
\usepackage{siunitx} % SI Units 
\usepackage{lscape} % Portrait- oder Landscape ausrichtung
\usepackage[absolute]{textpos} % Positionierung
\usepackage{amsmath}

% Schriftart Helvetica:
\usepackage[scaled]{helvet}
\renewcommand{\familydefault}{\sfdefault}

% TUM Schriftfarbe:
\usepackage{color} %Schriftfarbe
\definecolor{TUMblau}{RGB}{0, 82, 147}

\usepackage{calc} % Berechnungen
\usepackage{tabto} % Tabulatoren
\usepackage{parskip}

% Silbentrennung:
\usepackage{hyphenat}
\hyphenation{TUM in-te-res-siert} % Eigene Silbentrennung
%\tolerance 2414
%\hbadness 2414
%\emergencystretch 1.5em
%\hfuzz 0.3pt
%\widowpenalty=10000     % Hurenkinder
%\clubpenalty=10000      % Schusterjungen
%\vfuzz \hfuzz

\usepackage[hidelinks]{hyperref} % Hyperlinks
\usepackage[onehalfspacing]{setspace} % 1,5facher Zeilenabstand
\usepackage{calc} % Berechnungen
\usepackage{enumitem} % Mehr Kontrolle über itemize-, enumerate- und description-Umgebungen
\usepackage{relsize} % Schriftgröße in Abhängigkeit von aktueller anpassen
\usepackage{tabularx} % Flexiblere Tabellen
\usepackage{caption} % Anpassen von Beschriftungen

% Nummerierung von Abbildungen & Tabellen durchgängig, statt nach Kapiteln:
\usepackage{chngcntr}
\counterwithout{figure}{chapter}
\counterwithout{table}{chapter}

% Abkürzungen, Glossare:
\usepackage[%
    xindy,% xindy zum Indexieren verwenden
    acronym,% Separates Akronym-Verzeichnis
    nopostdot,% Kein Punkt am Ende einer Beschreibung im Glossar
    nomain
%    automake, 
%    nonumberlist
]{glossaries}

% Debugging:
%\usepackage{showframe} % Layout-Boxen anzeigen
%\usepackage{layout} % Layout-Informationen
%\usepackage{printlen} % Längenwerte ausgeben



%% Neue Pakete
\usepackage{subcaption}
\usepackage[style=ieee,natbib=true,dashed=false]{biblatex}
\bibliography{Literatur}
\usepackage{lipsum}
\sisetup{detect-all,
	per-mode=fraction,
%	parse-numbers=false
}
\DeclareSIUnit\particle{particle}
\DeclareSIUnit\Ferrum{Fe}
\DeclareSIUnit\sccm{sccm}
\DeclareSIUnit\torr{Torr}
\DeclareSIUnit[number-unit-product = {}]{\inch}{\textquotedbl}
\newcolumntype{R}{>{\raggedright\arraybackslash}X}
\newcolumntype{C}{>{\centering\arraybackslash}X} % Automatische Texttrennung in Tabellen
\usepackage{cancel}
\usepackage{listings}
\lstset{
language=Matlab,
showstringspaces = false}
\usepackage[titles]{tocloft}